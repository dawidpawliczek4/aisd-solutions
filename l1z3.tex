\documentclass[11pt,a4paper]{article}
\usepackage[utf8]{inputenc}
\usepackage[T1]{fontenc}
\usepackage[polish]{babel}
\usepackage{amsmath,amssymb}
\usepackage{algorithm}
\usepackage{algpseudocode}
\usepackage{geometry}
\geometry{margin=2.5cm}

\begin{document}

\begin{center}
\Large Dawid Pawliczek\\
Lista 1, Zadanie 3
\end{center}

\bigskip

\paragraph{Treść zadania.}
\emph{Porządkiem topologicznym} wierzchołków acyklicznego digrafu
$G=(V,E)$ nazywamy taki liniowy porządek jego wierzchołków, w którym
początek każdej krawędzi występuje przed jej końcem.
Jeśli wierzchołki z~$V$ utożsamimy z~kolejnymi liczbami naturalnymi
$1,2,\dots,|V|$, to każdy porządek liniowy możemy opisać
permutacją tych liczb, co pozwala również na \emph{leksykograficzne}
porównywanie porządków.
Ułóż algorytm, który dla zadanego acyklicznego digrafu
znajduje \textbf{pierwszy leksykograficznie} porządek topologiczny.

\bigskip
\paragraph{Pomysł (modyfikacja algorytmu Kahna).}
Klasyczne sortowanie topologiczne Kahna usuwa kolejno
wierzchołki o~zerowym stopniu wejściowym (\textit{indegree}).
Aby otrzymać pierwszą permutację leksykograficznie,
zawsze wybieramy \emph{najmniejszy etykietą} spośród wierzchołków
o~\textit{indegree}\;$=0$.
Do realizacji wystarczy kolejka priorytetowa (np.\ kopiec).

\bigskip
\paragraph{Algorytm.}

\begin{algorithm}[H]
\caption{\textsc{LexicoKahn}$(G=(V,E))$}
\begin{algorithmic}[1]
\State $Q\gets$ pusta min‐kolejka priorytetowa
\State $order\gets[\,]$ \Comment{wynikowa permutacja}
\ForAll{$v\in V$}
  \State oblicz $\text{indeg}[v]$
  \If{$\text{indeg}[v]=0$}
     \State $Q.\textsc{push}(v)$
  \Fi
\EndFor
\While{$Q\neq\varnothing$}
  \State $v\gets Q.\textsc{popMin}()$ \Comment{najmniejsza etykieta}
  \State $order.\textsc{append}(v)$
  \ForAll{następnik $u$ wierzchołka $v$}
     \State $\text{indeg}[u]\gets\text{indeg}[u]-1$
     \If{$\text{indeg}[u]=0$}
        \State $Q.\textsc{push}(u)$
     \Fi
  \EndFor
\EndWhile
\State \Return{$order$}
\end{algorithmic}
\end{algorithm}

\paragraph{Złożoność.}
Każdy wierzchołek jest dokładnie raz wstawiany i zdejmowany
z~kopca; operacje te kosztują $O(\log |V|)$.
Łączny czas: $O\bigl((|V|+|E|)\,\log |V|\bigr)$,
pamięć pomocnicza: $O(|V|)$.

\end{document}
