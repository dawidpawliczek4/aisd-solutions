\documentclass[11pt,a4paper]{article}
\usepackage[utf8]{inputenc}
\usepackage[T1]{fontenc}
\usepackage[polish]{babel}
\usepackage{amsmath,amssymb}
\usepackage{algorithm}
\usepackage{algpseudocode}
\usepackage{geometry}
\geometry{margin=2.5cm}

\begin{document}

\begin{center}
\Large Dawid Pawliczek\\
Lista 2, Zadanie 2
\end{center}

\bigskip
%%%%%%%%%%%%%%%%%%%%%%%%%%%%%%%%%%%%%%%%%%%%%%%%%%%%%%%%%%%%%%%%%%%%
\section*{Treść zadania}

Danych jest $n$ odcinków
\(
  I_j=\langle p_j,k_j\rangle
\)
leżących na osi $OX$, $j=1,\dots,n$.
Ułożyć algorytm znajdujący zbiór
\(S\subseteq\{I_1,\dots,I_n\}\)
nieprzecinających się odcinków
o~największej mocy~\(|S|\).

%%%%%%%%%%%%%%%%%%%%%%%%%%%%%%%%%%%%%%%%%%%%%%%%%%%%%%%%%%%%%%%%%%%%
\section*{Intuicja}

\paragraph{Dynamiczne~DP (pomysł odrzucony).}
Można próbować rozwiązać problem
przez DP: \(\text{dp}[i][j]\) –
maksymalna liczba odcinków mieszczących się
w przedziale \([i,j]\).
Wzór rekurencyjny:
\[
  \text{dp}[i][j]=
  \max_{\substack{I=\langle k,k'\rangle\\i\le k<k'\le j}}
  \bigl(
     \text{dp}[i][k]+1+\text{dp}[k'][j]
  \bigr),
\]
jednak prowadzi to do kosztu kwadratowego.

\paragraph{Kluczowa obserwacja (wymiana).}
W \emph{dowolnym} optymalnym rozwiązaniu
można wybrać odcinek, który kończy się
\emph{najwcześniej} spośród wszystkich.
Jeśli optymalne rozwiązanie tego nie robi,
zamieniamy jego pierwszy odcinek
na odcinek o~najmniejszym~$k$:
liczność zbioru nie spada,
więc otrzymujemy równorzędne optimum.

Stąd wynika zachłanna konstrukcja:
wybieramy najwcześniej kończący się odcinek,
wyrzucamy wszystkie z nim kolidujące
i powtarzamy na pozostałych.

%%%%%%%%%%%%%%%%%%%%%%%%%%%%%%%%%%%%%%%%%%%%%%%%%%%%%%%%%%%%%%%%%%%%
\section*{Algorytm zachłanny}

\begin{algorithm}[H]
\caption{\textsc{BestIntervals}}
\begin{algorithmic}[1]
\Require{zbiór odcinków $I=\{I_j=\langle p_j,k_j\rangle\}$}
\Ensure{maksymalny zbiór rozłącznych odcinków}
\State posortuj odcinki niemalejąco po~$k_j$
\State $S\gets[\,]$;\quad$lastEnd\gets-\infty$
\ForAll{$I_j=\langle p_j,k_j\rangle$ w~tej kolejności}
  \If{$p_j\ge lastEnd$}
     \State dołącz $I_j$ do $S$
     \State $lastEnd\gets k_j$
  \EndIf
\EndFor
\State \Return{$S$}
\end{algorithmic}
\end{algorithm}

%%%%%%%%%%%%%%%%%%%%%%%%%%%%%%%%%%%%%%%%%%%%%%%%%%%%%%%%%%%%%%%%%%%%
\section*{Dowód poprawności}

Dowód przez \emph{wymianę}.
Niech $S_g$ – zbiór zwrócony przez algorytm,
$S^\star$ – dowolne optymalne rozwiązanie,
a odcinki obu zbiorów
posortowane rosnąco po końcach:
\(S_g=(g_1,g_2,\dots)\), \(S^\star=(s_1,s_2,\dots)\).

\begin{enumerate}
\item
Pierwszy odcinek algorytmu, $g_1$,
ma najmniejszy koniec spośród wszystkich,
więc $k(g_1)\le k(s_1)$.
Jeżeli $g_1\ne s_1$,
konstruujemy zbiór $S'=(g_1,s_2,s_3,\dots)$.
Ponieważ $p(s_2)\ge k(s_1)\ge k(g_1)$,
$g_1$ nie przecina $s_2$,
zaś liczność $|S'|=|S^\star|$,
więc $S'$ także jest optymalny.

\item
Stosując tę samą operację wymiany
indukcyjnie dla kolejnych pozycji,
otrzymujemy optymalne rozwiązanie,
które zaczyna się prefiksem $g_1,g_2,\dots$.
Po skończeniu indukcji całe $S_g$
musi być optymalne,
czyli $|S_g|=|S^\star|$.
\end{enumerate}
\hfill$\square$

%%%%%%%%%%%%%%%%%%%%%%%%%%%%%%%%%%%%%%%%%%%%%%%%%%%%%%%%%%%%%%%%%%%%
\section*{Złożoność}

Sortowanie – $O(n\log n)$,
pętla – $O(n)$.
Łącznie \(\;O(n\log n)\).
Pamięć pomocnicza \(O(1)\)
(poza~tablicą wejściową).

\end{document}
