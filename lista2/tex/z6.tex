\documentclass[11pt,a4paper]{article}
\usepackage[utf8]{inputenc}
\usepackage[T1]{fontenc}
\usepackage[polish]{babel}
\usepackage{amsmath,amssymb}
\usepackage{algorithm}
\usepackage{algpseudocode}
\usepackage{geometry}
\geometry{margin=2.5cm}

\begin{document}

\begin{center}
\Large Dawid Pawliczek\\
Lista 2, Zadanie 6
\end{center}

\bigskip
%%%%%%%%%%%%%%%%%%%%%%%%%%%%%%%%%%%%%%%%%%%%%%%%%%%%%%%%%%%%%%%%%%%%
\section*{Treść}

Dane jest $n$-wierzchołkowe drzewo $T=(V,E)$
oraz liczba całkowita $k\ge1$.
Trzeba  \textbf{p}-pokolorować możliwie wiele wierzchołków,
tak aby \emph{na każdej prostej ścieżce} znajdowało się
\emph{co najwyżej $k$} p-pokolorowanych wierzchołków.

%%%%%%%%%%%%%%%%%%%%%%%%%%%%%%%%%%%%%%%%%%%%%%%%%%%%%%%%%%%%%%%%%%%%
\section*{Intuicja}

Oznaczmy przez \emph{liść} wierzchołek stopnia~1.
Każda prosta ścieżka w~drzewie ma dwa końce,
które możemy \emph{rozszerzyć} do liści.
Jeśli będziemy kolorować wyłącznie liście,
a następnie usuwać je z drzewa i
powtarzać tę operację,  
to w~jednym ciągu usuwania
\emph{obydwa} końce dowolnej ścieżki
otrzymają p-kolor najwyżej raz.
Wykonując tę procedurę
\(
  \lfloor k/2\rfloor
\)
razy
zabezpieczamy się,
że na każdej ścieżce znajdą się co najwyżej
$k$ p-wierzchołków.
Gdy $k$ jest nieparzyste,
można dodatkowo pokolorować
\emph{dowolny} jeszcze niepokolorowany wierzchołek.

%%%%%%%%%%%%%%%%%%%%%%%%%%%%%%%%%%%%%%%%%%%%%%%%%%%%%%%%%%%%%%%%%%%%
\section*{Algorytm}

\begin{algorithm}[H]
\caption{\textsc{MaxPColour}$(T,k)$}
\begin{algorithmic}[1]
\Require{drzewo $T=(V,E)$, liczba $k$}
\Ensure{zbiór p-pokolorowanych wierzchołków}
\State \textbf{for} $i\gets1$ \textbf{to} $\lfloor k/2\rfloor$ \textbf{do}
   \State\hspace{1em} $L\gets$ wszystkie liście bieżącego $T$
   \State\hspace{1em} pokoloruj każdy $v\in L$
   \State\hspace{1em} usuń wierzchołki $L$ z~$T$
\If{$k$ jest nieparzyste}
   \State pokoloruj dowolny niepokolorowany $v\in V$
\EndIf
\State \Return{zaznaczony zbiór}
\end{algorithmic}
\end{algorithm}

%%%%%%%%%%%%%%%%%%%%%%%%%%%%%%%%%%%%%%%%%%%%%%%%%%%%%%%%%%%%%%%%%%%%
\section*{Poprawność}

\begin{itemize}
\item
\textbf{Górne ograniczenie.}
Po każdej iteracji usuwania liści długość
dowolnej ścieżki skraca się o co najmniej
dwa wierzchołki (oba końce).
Stąd ścieżka może zawierać
co najwyżej
\(
  2\cdot\lfloor k/2\rfloor
  + (k\bmod 2) = k
\)
p-pokolorowanych wierzchołków.
\item
\textbf{Maksymalność.}
Na drodze od każdego liścia w~głąb drzewa
pokryliśmy dokładnie
$\lfloor k/2\rfloor$ poziomów,
więc dołożenie \emph{jakiegokolwiek}
kolejnego liścia przekroczyłoby limit $k$
na ścieżce łączącej dwa liście.
Dodatkowy wierzchołek (gdy $k$ nieparzyste)
te limit~$k$ dokładnie domyka.
\end{itemize}
Zatem algorytm koloruje optymalną liczbę
wierzchołków.
\hfill$\square$

%%%%%%%%%%%%%%%%%%%%%%%%%%%%%%%%%%%%%%%%%%%%%%%%%%%%%%%%%%%%%%%%%%%%
\section*{Złożoność}

Każdy wierzchołek jest usuwany najwyżej raz.
Łączny czas i~pamięć wynoszą
\(
  O(n).
\)

\end{document}
