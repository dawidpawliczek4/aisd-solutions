\documentclass[11pt,a4paper]{article}
\usepackage[utf8]{inputenc}
\usepackage[T1]{fontenc}
\usepackage[polish]{babel}
\usepackage{amsmath,amssymb}
\usepackage{algorithm}
\usepackage{algpseudocode}
\usepackage{geometry}
\geometry{margin=2.5cm}

\begin{document}

\begin{center}
\Large Dawid Pawliczek\\
Lista 2, Zadanie 5
\end{center}

\bigskip
%%%%%%%%%%%%%%%%%%%%%%%%%%%%%%%%%%%%%%%%%%%%%%%%%%%%%%%%%%%%%%%%%%%%
\section*{Treść}

Dany jest graf $G=(V,E)$ oraz liczba całkowita $k$.
Znaleźć możliwie największy podzbiór
$V'\subseteq V$ taki, że dla każdego $v\in V'$
\[
\bigl|\{u\in V':\{u,v\}\in E\}\bigr|\;\ge\;k
\quad\text{i}\quad
\bigl|\{u\in V':\{u,v\}\notin E\}\bigr|\;\ge\;k.
\]

Zapisując $d_{V'}(v)$~— stopień $v$ wewnątrz $V'$,
warunek można skrócić do
\[
k\;\le\;d_{V'}(v)\;\le\;(|V'|-1)-k.
\]

%%%%%%%%%%%%%%%%%%%%%%%%%%%%%%%%%%%%%%%%%%%%%%%%%%%%%%%%%%%%%%%%%%%%
\section*{Algorytm „przycinania”}

\begin{algorithm}[H]
\caption{\textsc{LargestSubset}$(G,k)$}
\begin{algorithmic}[1]
\State $V'\gets V$; \quad kolejka $Q\gets$ wszystkie $v\in V'$
\While{$Q\neq\varnothing$}
   \State zdejmij $v$ z $Q$
   \State \textbf{if } $d_{V'}(v)<k$ \textbf{or}
          $d_{V'}(v) > |V'|-1-k$
          \textbf{then}
          usuń $v$ z $V'$ i dołóż wszystkich sąsiadów/niesąsiadów
          $v$ do~$Q$ (bo ich stopnie mogły się zmienić)
\EndWhile
\State \Return{$V'$}
\end{algorithmic}
\end{algorithm}

Idea: \emph{usuwamy każdą sprzeczną z~warunkiem $(k,k)$
końcówkę — aż do ustalenia się zbioru.}

%%%%%%%%%%%%%%%%%%%%%%%%%%%%%%%%%%%%%%%%%%%%%%%%%%%%%%%%%%%%%%%%%%%%
\section*{Dowód poprawności}

\begin{description}
\item[1. Zatrzymanie.]
   Gdy algorytm kończy, każdy węzeł $v\in V'$
   spełnia
   \(
     k\le d_{V'}(v)\le |V'|-1-k,
   \)
   więc $V'$ jest \emph{poprawne}.

\item[2. Nietracenie kandydatów.]
   Rozważ węzeł $v$ usuwany w pewnym kroku,
   gdy aktualny zbiór ma rozmiar $s=|V'|$ i
   stopień $d=d_{V'}(v)$.

   \begin{itemize}
   \item Gdy $d<k$,
         dalsze kroki kasują \emph{tylko} wierzchołki,
         więc stopień $v$ mógłby już
         \underline{tylko spadać}.
         Warunku $d\ge k$ nie da się odtworzyć.

   \item Gdy $d> s-1-k$,
         liczba niesąsiadów $v$ wynosi
         $s-1-d<k$.
         Usuwając jakiekolwiek wierzchołki:
         \[
           (s-1)\!-\!d \longrightarrow (s-2)\!-\!(d-1)
           =(s-1-d)\;<k,
         \]
         więc wciąż $mniej$ niż $k$.
         Warunku także nie da się przywrócić.
   \end{itemize}
   W obu przypadkach $v$ nie może należeć
   do \emph{żadnego} poprawnego nadzbioru
   obecnego $V'$.  Usuwanie jest zatem bezpieczne.

\item[3. Maksymalność.]
   Dowolny poprawny zbiór $S$
   jest podzbiorem zbioru operacyjnego
   na każdym etapie pętli,
   bo usuwamy wyłącznie węzły,
   które w $S$ znajdować się nie mogą
   (punkt~2).
   Po zakończeniu algorytmu
   zachodzi $S\subseteq V'$,
   więc rozmiar $V'$ jest
   \emph{co najmniej} tak duży, jak największy $S$.
\end{description}

Zatem zwrócony $V'$ jest największym
możliwym podzbiorem spełniającym wymaganie.
\hfill$\square$

%%%%%%%%%%%%%%%%%%%%%%%%%%%%%%%%%%%%%%%%%%%%%%%%%%%%%%%%%%%%%%%%%%%%
\section*{Złożoność}

Każdy wierzchołek trafia do kolejki
co najwyżej tyle razy, ilu ma sąsiadów,
więc łączny koszt to
\[
  O\bigl(|V|+|E|\bigr)
\]
przy wykorzystaniu prostych liczników stopni.

\end{document}
