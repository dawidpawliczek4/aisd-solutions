\documentclass[11pt,a4paper]{article}
\usepackage[utf8]{inputenc}
\usepackage[T1]{fontenc}
\usepackage[polish]{babel}
\usepackage{amsmath,amssymb}
\usepackage{algorithm}
\usepackage{algpseudocode}
\usepackage{geometry}
\geometry{margin=2.5cm}

\begin{document}

\begin{center}
\Large Dawid Pawliczek\\
Lista 2, Zadanie 4
\end{center}

\bigskip
\section*{Treść zadania}
Masz początkowo do dyspozycji $m$ monet o nominale $1$ i nieskończoną
liczbę monet o nominale $100$.
W kolejnych $n$ dniach robisz zakupy
(w $i$-tym dniu za kwotę $c_i$).
Jeśli nie odliczysz dokładnie $c_i$ w jedynkach,
kasa wyda resztę minimalną liczbą monet,
używając wyłącznie nominałów $1$ i $100$,
a Twoja „atrakcyjność kliencka” zmniejszy się
o~$x\cdot w_i$, gdzie $x$ to liczba wydanych monet,
zaś $w_i$ to zadany współczynnik.
(Zakładamy $1\le c_i,w_i\le n$).
Wyznacz, o~ile \emph{co najmniej} spadnie
łączna atrakcyjność po wszystkich zakupach.

\bigskip
\section*{Rozwiązanie dynamiczne}

Niech
\[
  \text{dp}[i][m]\;=\;
  \text{największa możliwa atrakcyjność po } i\text{ dniach}
  \text{ przy stanie portfela } m.
\]
Dla uproszczenia przyjmujemy $c_i<100$,
bo zawsze w pierwszej kolejności płacimy pełnymi setkami.

\begin{align*}
\text{dp}[i+1][m-c_i]
  &=\max\bigl(\text{dp}[i+1][m-c_i],\,\text{dp}[i][m]\bigr),
  &&\text{jeśli }c_i\le m;\\[4pt]
\text{dp}[i+1][m+100-c_i]
  &=\max\bigl(\text{dp}[i+1][m+100-c_i],
              \,\text{dp}[i][m]-(100-c_i)w_i\bigr),
  &&\text{jeśli }c_i>m.
\end{align*}

Liczba możliwych stanów rośnie najwyżej
o~$100$ dziennie, więc
złożoność wynosi $O(n^2)$.

\bigskip
\section*{Obserwacja prowadząca do rozwiązania zachłannego}

W każdym dniu wybór sprowadza się do:
\[
  \boxed{\text{zapłacić dokładnie}} 
  \quad\text{lub}\quad
  \boxed{\text{zapłacić }100\text{ i przyjąć koszt }(100-c_i)w_i.}
\]
Jeżeli w $i$-tym dniu zabraknie jedynek,
możemy „cofnąć” którąś z wcześniejszych decyzji
(zamienić płatność dokładną na setkę),
ale warto to zrobić tylko tam,
gdzie koszt $(100-c_j)w_j$ jest najmniejszy.

\bigskip
\section*{Algorytm zachłanny}

\begin{algorithm}[H]
\caption{\textsc{GreedyMin}}
\begin{algorithmic}[1]
\Require{$m$, tablice $c[1\dots n]$, $w[1\dots n]$}
\Ensure{łączny spadek atrakcyjności}
\State $loss\gets0$, \quad $heap\gets$ min-kopiec
\For{$i\gets1$ \textbf{to} $n$}
   \State $m\gets m-c_i$ \Comment{próbujemy zapłacić dokładnie}
   \State $heap.\textsc{push}\bigl((100-c_i)w_i\bigr)$
      \State $loss\gets loss+heap.\textsc{popMin}()$ \Comment{najtańsza „cofka”}
      \State $m\gets m+100$
\EndFor
\State \Return{$loss$}
\end{algorithmic}
\end{algorithm}

\medskip
\noindent
\textbf{Złożoność.}
Każdy element wstawiamy i usuwamy z kopca co najwyżej raz,
stąd czas $O(n\log n)$, pamięć $O(n)$.

\medskip
\noindent
%%%%%%%%%%%%%%%%%%%%%%%%%%%%%%%%%%%%%%%%%%%%%%%%%%%%%%%%%%%%%%%%%%%%
%%%%%%%%%%%%%%%%%%%%%%%%%%%%%%%%%%%%%%%%%%%%%%%%%%%%%%%%%%%%%%%%%%%%
\section*{Poprawność}

\paragraph{Inwariant.}
Po obsłużeniu pierwszych $i$ dni algorytm utrzymuje:
\begin{enumerate}
  \item portfel zawiera co najmniej $0$ monet $1$,
  \item z wszystkich dotychczas rozważonych dni do zbioru
        płatności $100$ wybrano \emph{najtańsze} (koszt
        $(100-c_t)w_t$) tak, aby liczba jedynek nie była ujemna.
\end{enumerate}

Punkt (1) jest spełniany, bo w razie deficytu
algorytm natychmiast „cofa” kolejne dni,
dopisując po $100$ jednostek do stanu portfela,
aż $m\ge0$.
Punkt (2) zachodzi, gdyż cofamy zawsze
\emph{najmniejszą} dostępną wartość z kopca.

\paragraph{Greedy $\to$ optymalny.}
Załóżmy, że istnieje rozwiązanie $OPT$
o mniejszym łącznym koszcie od wyniku algorytmu.
Prześledźmy dni w kolejności
i wybierzmy pierwszy dzień, w którym
zbiory „cofnięć” algorytmu i $OPT$ różnią się.
W tym momencie algorytm wybrał
koszt minimalny spośród wszystkich
niedotychczasowych dni,
a $OPT$ — koszt co najmniej tak duży.
Zamieniając w $OPT$
jego droższy dzień na tańszy dzień algorytmu,
poprawiamy lub nie zmieniamy kosztu i
\textbf{nie} psujemy stanu portfela
(bo liczba cofnięć się nie zmniejsza).
Wykonując taką operację każdorazowo,
stopniowo przekształcamy $OPT$
w rozwiązanie identyczne z wynikiem algorytmu,
bez zwiększania ceny.
Sprzeczność z założeniem, że algorytm był droższy.

\paragraph{Konkluzja.}
Zawsze, gdy brakuje jedynek, trzeba dołożyć
co najmniej jedną płatność $100$,
a w każdym takim momencie algorytm wybiera
najtańszą możliwą opcję —
dlatego łączny koszt nie może być mniejszy.
\hfill$\square$

\end{document}
