\documentclass[11pt,a4paper]{article}
\usepackage[utf8]{inputenc}
\usepackage[T1]{fontenc}
\usepackage[polish]{babel}
\usepackage{amsmath,amssymb}
\usepackage{algorithm}
\usepackage{algpseudocode}
\usepackage{geometry}
\geometry{margin=2.5cm}

\begin{document}

\begin{center}
\Large Dawid Pawliczek\\
Notatka -- \textit{Longest Increasing Subsequence}
\end{center}

\bigskip
%%%%%%%%%%%%%%%%%%%%%%%%%%%%%%%%%%%%%%%%%%%%%%%%%%%%%%%%%%%%%%%%%%%%
\section*{Treść}

Dla danej tablicy $A[1\dots n]$ należy znaleźć
\emph{najdłuższy ściśle rosnący podciąg}
(nie koniecznie spójny!) jej elementów.

%%%%%%%%%%%%%%%%%%%%%%%%%%%%%%%%%%%%%%%%%%%%%%%%%%%%%%%%%%%%%%%%%%%%
\section*{Rozwiązanie $O(n^2)$ – dynamiczne}

\paragraph{Definicja stanu.}
\[
  \text{dp}[i] \;=\;
  \text{długość LIS kończącego się dokładnie na pozycji } i.
\]
\[
  \text{dp}[i]=
    1 + \max_{\substack{j<i\\A[j]<A[i]}}\text{dp}[j],
  \quad
  \text{dp}[i]=1 \text{ jeśli brak takiego } j.
\]

\begin{algorithm}[H]
\caption{\textsc{NaiveLIS}}
\begin{algorithmic}[1]
\Require{$A[1\dots n]$}
\Ensure{długość LIS i jeden taki podciąg}
\State \textbf{for} $i\gets1$ \textbf{to} $n$:
   \State\hspace{1em}$\text{dp}[i]\gets1$, $\text{prev}[i]\gets0$
   \State\hspace{1em}\textbf{for} $j\gets1$ \textbf{to} $i-1$:
      \If{$A[j]<A[i]$ \textbf{and} $\text{dp}[j]+1>\text{dp}[i]$}
         \State\hspace{2em}$\text{dp}[i]\gets\text{dp}[j]+1$
         \State\hspace{2em}$\text{prev}[i]\gets j$
      \EndIf
\State $\text{best}\gets\arg\max_i\text{dp}[i]$
\State odtwarzaj ciąg cofając się po wskaźnikach \texttt{prev}
\State \Return{($\text{dp}[\text{best}]$, ciąg)}
\end{algorithmic}
\end{algorithm}

\paragraph{Odzyskiwanie podciągu.}
Tablica \texttt{prev} wskazuje poprzedni wybrany indeks.
Startujemy od \texttt{best} i idziemy wstecz,
odwracając otrzymaną listę – otrzymujemy poprawny LIS.

%%%%%%%%%%%%%%%%%%%%%%%%%%%%%%%%%%%%%%%%%%%%%%%%%%%%%%%%%%%%%%%%%%%%
\section*{Rozwiązanie $O(n\log n)$ – patience sorting}

\paragraph{Idea.}
Utrzymujemy tablicę \(
  \text{tail}[l]
\) –
najmniejszą możliwą \emph{wartość końcową}
rosnącego podciągu długości $l$.
Wektor \texttt{tail} jest rosnący, więc
aktualizacje wykonujemy wyszukiwaniem binarnym.

\paragraph{Algorytm.}
\begin{algorithm}[H]
\caption{\textsc{FastLIS}}
\begin{algorithmic}[1]
\Require{$A[1\dots n]$}
\Ensure{długość LIS i jeden taki podciąg}
\State $\text{tail}[0]\gets-\infty$;\;
       $\text{pos}[0]\gets0$     \Comment{wartość i pozycja końca}
\State $\text{len}\gets0$
\For{$i\gets1$ \textbf{to} $n$}
   \State $l\gets$ najmniejsze $p\in[1,\text{len}]$
          z $\text{tail}[p]\ge A[i]$
          \textbf{(binarne szukanie)}  
   \If{nie znaleziono} $l\gets\text{len}+1$
   \State $\text{tail}[l]\gets A[i]$; \,
          $\text{pos}[l]\gets i$      \Comment{aktualizacja wartości}
   \State $\text{pred}[i]\gets\text{pos}[l-1]$ \Comment{poprzednik w LIS}
   \State $\text{len}\gets\max(\text{len},\,l)$
\EndFor
\State
\begin{minipage}[t]{0.9\linewidth}
\textbf{rekonstrukcja:}\\
$k\gets\text{pos}[\text{len}]$;\quad
\textbf{while} $k\neq0$:
  wypisz $A[k]$,\, $k\gets\text{pred}[k]$\\
odwróć otrzymaną listę
\end{minipage}
\State \Return{($\text{len}$, ciąg w poprawnej kolejności)}
\end{algorithmic}
\end{algorithm}

\paragraph{Dlaczego działa?}
Jeśli mamy dwa podciągi tej samej długości,
ten z mniejszym elementem końcowym jest lepszy,
bo daje większą szansę przedłużenia.
Stąd zawsze warto nadpisać
pierwszy element w \texttt{tail}
\emph{większy równy} od $A[i]$.

\paragraph{Rekonstrukcja.}
\texttt{pos} zapamiętuje indeks w tablicy $A$,
na którym dany ogon \texttt{tail} powstał.
\texttt{pred}[i] wskazuje poprzedni element
podciągu kończącego się w $A[i]$.
Śledząc wskazania od ostatniego indeksu
podciągu długości \texttt{len}
otrzymujemy LIS w~odwrotnym porządku.

%%%%%%%%%%%%%%%%%%%%%%%%%%%%%%%%%%%%%%%%%%%%%%%%%%%%%%%%%%%%%%%%%%%%
\section*{Złożoność}

\[
  O(n^2) \text{ — wersja DP},
  \qquad
  O(n\log n) \text{ — wersja szybka}
\]
z pamięcią $O(n)$ (obie metody).

\end{document}
