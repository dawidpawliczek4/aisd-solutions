\documentclass[11pt,a4paper]{article}
\usepackage[utf8]{inputenc}
\usepackage[T1]{fontenc}
\usepackage[polish]{babel}
\usepackage{amsmath,amssymb}
\usepackage{algorithm}
\usepackage{algpseudocode}
\usepackage{geometry}
\geometry{margin=2.5cm}

\begin{document}

\begin{center}
\Large Dawid Pawliczek\\
Lista 3, Zadanie 4
\end{center}

\bigskip
%%%%%%%%%%%%%%%%%%%%%%%%%%%%%%%%%%%%%%%%%%%%%%%%%%%%%%%%%%%%%%%%%%%%
\section*{Treść}

Dana jest kolekcja $n$ prostych
\(
  l_i :\; y=a_i x + b_i
  \;(i=1,\dots,n)
\)
na płaszczyźnie, przy czym żadne trzy nie przecinają się
w~jednym punkcie.
Mówimy, że prosta $l_i$ jest \emph{widoczna}
z~punktu $(0,+\infty)$,
jeśli istnieje punkt $q\in l_i$ taki,
że odcinek $(0,+\infty)\,q$
nie przecina żadnej innej prostej (oprócz, ewentualnie, w~punkcie~$q$).
Zaprojektować algorytm znajdujący wszystkie widoczne proste.

Geometria patrzenia „z~góry” sprowadza problem
do wyznaczenia \emph{górnej obwiedni} układu prostych.

%%%%%%%%%%%%%%%%%%%%%%%%%%%%%%%%%%%%%%%%%%%%%%%%%%%%%%%%%%%%%%%%%%%%
\section*{Rozwiązanie}

\subsection*{1. Wstępne sortowanie}

\begin{itemize}
\item
Posortuj proste rosnąco według współczynnika kierunkowego $a$.
\item
Jeżeli dwie proste mają ten sam $a$, pozostaw tylko tę
o większym~$b$ (druga leży \emph{niżej},
więc nigdy nie będzie widoczna).
\end{itemize}

\subsection*{2. Budowa obwiedni stosem}

Przechodzimy po posortowanej liście.
Stos $S$ przechowuje zawsze aktualny zestaw widocznych prostych
w~kolejności rosnących $a$.

\begin{algorithm}[H]
\caption{\textsc{VisibleLines}}
\begin{algorithmic}[1]
\Require{lista $L$ posortowana wg $a$ (po kroku 1)}
\Ensure{zbiór linii widocznych z $(0,+\infty)$}
\State $S\gets$ pusty stos
\ForAll{$l$ w~$L$}                           \Comment{$l:y=ax+b$}
   \While{$|S|\ge2$}
      \State niech $l_1$ = $S\text{.top()}$, $l_2$ = drugi od góry w $S$
      \State wyznacz $x$‐owe współrzędne punktów
             przecięcia $l_1\cap l_2$ oraz $l_2\cap l$
      \If{${\rm x}(l_1\cap l_2)\;\ge\;{\rm x}(l_2\cap l)$}
         \State $S.\text{pop()}$  \Comment{$l_1$ zasłonięty — usuń}
      \Else \textbf{break}
      \EndIf
   \EndWhile
   \State $S.\text{push}(l)$
\EndFor
\State \Return{$S$}
\end{algorithmic}
\end{algorithm}

\paragraph{Komentarz.}
Jeśli punkt przecięcia nowej prostej $l$
z poprzednikiem $l_2$ leży \emph{na lewo}
od przecięcia $l_2$ z~$l_1$,
to $l_1$ znajduje się całkowicie pod obwiednią
i znika z~widoku; pętla \texttt{while} usuwa
kolejne takie linie aż warunek przestanie być spełniony.

%%%%%%%%%%%%%%%%%%%%%%%%%%%%%%%%%%%%%%%%%%%%%%%%%%%%%%%%%%%%%%%%%%%%
%%%%%%%%%%%%%%%%%%%%%%%%%%%%%%%%%%%%%%%%%%%%%%%%%%%%%%%%%%%%%%%%%%%%
\section*{Poprawność}

\begin{enumerate}
\item \textbf{(Usuwanie nadmiarowych linii o tym samym $a$).}  
      Jeśli $a_i=a_j$ i $b_i<b_j$, linia $l_i$ leży pod $l_j$ dla
      każdego $x$; z~punktu $(0,+\infty)$ nigdy nie może być zatem
      widoczna.  Jej odrzucenie nie wpływa na wynik.

\item \textbf{(Monotoniczność przecięć).}  
      Niech stos zawiera (od dołu) linie
      $l_0,l_1,\dots,l_t$ w kolejności rosnących współczynników $a$.  
      Łatwo sprawdzić, że
      \[
        x\bigl(l_{t-1}\!\cap l_t\bigr)
        \;<\;
        x\bigl(l_{t}\!\cap l\bigr)
      \]
      oznacza, iż $l_t$ i $l$ przecinają się \emph{na prawo} od
      przecięcia $l_{t-1}$ z $l_t$.  
      Odcinek $l_tl$ wypukłej obwiedni pozostaje więc odkryty; $l_t$
      musi zostać w odpowiedzi, a pętla \texttt{while} się kończy.

\item \textbf{(Warunek usuwania).}  
      Jeśli natomiast
      $
        x(l_{t-1}\!\cap l_t)
        \ge
        x(l_t\!\cap l),
      $
      cała prosta $l_t$ leży pod odcinkiem obwiedni
      łączącym $l_{t-1}$ z $l$.  
      Z~punktu $(0,+\infty)$ $l_t$ jest całkowicie zasłonięta,
      więc jej usunięcie zachowuje zbiór linii widocznych.
      Utrwalenie tej właściwości dla kolejnych
      linii zapewnia, że stos zawiera
      wyłącznie krawędzie końcowej obwiedni.

\item \textbf{(Zużycie każdej linii najwyżej raz).}  
      Każda linia trafia na stos raz
      i z niego schodzi najwyżej raz,
      nie może więc zostać „nadmiernie” odsiana.
\end{enumerate}

Po przetworzeniu całego wejścia
stos zawiera dokładnie te i tylko te proste,
które tworzą górną obwiednię,
a więc są widoczne z~punktu $(0,+\infty)$.
\hfill$\square$


%%%%%%%%%%%%%%%%%%%%%%%%%%%%%%%%%%%%%%%%%%%%%%%%%%%%%%%%%%%%%%%%%%%%
\section*{Złożoność}

\begin{itemize}
\item
Sortowanie po współczynniku $a$: \;\;$O(n\log n)$.
\item
Pętla główna: każde wstawienie lub usunięcie
to koszt $O(1)$, a każda linia
może być wypchnięta ze stosu najwyżej raz –
łącznie $O(n)$.
\end{itemize}

\[
\boxed{\;T(n)=O(n\log n)\;}, \qquad
\boxed{\;\text{pamięć } O(n)\;}
\]

\end{document}
