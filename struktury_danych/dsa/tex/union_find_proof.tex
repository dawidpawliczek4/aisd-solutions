\documentclass[11pt,a4paper]{article}
\usepackage[utf8]{inputenc}
\usepackage[T1]{fontenc}
\usepackage[polish]{babel}
\usepackage{amsmath,amssymb,amsthm}
\usepackage{geometry}
\geometry{margin=2.5cm}

\begin{document}

\begin{center}
\Large Dawid Pawliczek\\
\vspace{2mm}
\textbf{Union–Find}\\[1ex]
Dowód, że koszt zamortyzowany wynosi $O\!\bigl(n\log^{\!*}\!n\bigr)$
\end{center}
\bigskip

%%%%%%%%%%%%%%%%%%%%%%%%%%%%%%%%%%%%%%%%%%%%%%%%%%%%%%%%%%%%%%%%%%%%
\section*{Definicje}

\begin{itemize}
  \item Każdy węzeł $x$ ma atrybut \emph{rank}\,$(x)$.
  \item \texttt{makeSet}$(x)$ ustawia $rank(x)=0$ i $parent(x)=x$.
  \item \texttt{union}$(x,y)$~łączy dwa zbiory:
        korzeniem zostaje ten, którego ranga jest większa;
        przy rangach równych wybieramy dowolny i
        \emph{zwiększamy} jego rangę o~1.
  \item \texttt{find}$(x)$ zwraca korzeń zbioru
        oraz ustawia dla każdego odwiedzonego węzła
        $parent=\text{korzeń}$ (kompresja ścieżek).
\end{itemize}

%%%%%%%%%%%%%%%%%%%%%%%%%%%%%%%%%%%%%%%%%%%%%%%%%%%%%%%%%%%%%%%%%%%%
\section*{Lemat 1}

\textbf{Twierdzenie.}
\emph{Wysokość każdego drzewa jest nie większa niż ranga jego korzenia.}

\smallskip
\textbf{Dowód.}
Rangę zwiększamy tylko wtedy, gdy łączymy \emph{dwa} drzewa
o~tej samej randze $r$, wybierając jeden korzeń
i nadając mu rangę $r+1$.
Wysokość nowego drzewa rośnie wówczas najwyżej o~1,
czyli również maksymalnie do $r+1$.
Kroki indukcyjne zachowują tezę, a dla rangi~0 jest ona oczywista.
\hfill$\square$

\medskip
Uwaga: kompresja ścieżek może zmniejszyć wysokość drzewa,
dlatego nierówność nie działa w drugą stronę.

%%%%%%%%%%%%%%%%%%%%%%%%%%%%%%%%%%%%%%%%%%%%%%%%%%%%%%%%%%%%%%%%%%%%
\section*{Lemat 2}

\textbf{Twierdzenie.}
\emph{Dla każdego węzła $x$~~$rank(x)\le\lfloor\log_2 n\rfloor$.}

\smallskip
\textbf{Dowód.}
Aby ranga węzła osiągnęła wartość $r$,
musiało się połączyć co najmniej $2^{r}$ elementów:
każdy wzrost rangi wymaga scalenia
dwóch drzew o~tej samej randze.  
Zatem $2^{r}\le n$, skąd $r\le\log_2 n$.
\hfill$\square$

%%%%%%%%%%%%%%%%%%%%%%%%%%%%%%%%%%%%%%%%%%%%%%%%%%%%%%%%%%%%%%%%%%%%
\section*{Grupowanie rang a~$\log^{\!*}$}

Zamiast śledzić każdy \emph{pojedynczy} wzrost rangi (jest ich
$\le\log_2 n$), grupujemy je w~\textbf{poziomy}.
Definiujemy
\[
F(0)=1,\quad F(i)=2^{F(i-1)}\;(i\ge1).
\]
Poziom~$i$ zawiera wszystkie rangi
\(
F(i-1) < r \le F(i).
\)
Liczba poziomów potrzebnych,
by objąć maksymalną rangę $\le\log_2 n$,
wynosi $\log^{\!*}\!n$, ponieważ $F\bigl(\log^{\!*}\!n\bigr)\ge n$.

%%%%%%%%%%%%%%%%%%%%%%%%%%%%%%%%%%%%%%%%%%%%%%%%%%%%%%%%%%%%%%%%%%%%
\section*{Ile razy węzeł może zmienić rodzica?}

Podczas \texttt{find} każdy odwiedzony węzeł $v$ ustawia
\(
 parent(v)\leftarrow\text{korzeń}
\)
czyli na węzeł o \emph{ściśle większej} randze.
Węzeł może więc „awansować” co najwyżej:
raz z poziomu~0 do~1, raz z~1 do~2, \dots\
Łącznie najwyżej $\log^{\!*}\!n$ razy.

%%%%%%%%%%%%%%%%%%%%%%%%%%%%%%%%%%%%%%%%%%%%%%%%%%%%%%%%%%%%%%%%%%%%
\section*{Koszt operacji}

\begin{itemize}
  \item \texttt{union} -- koszt $O(1)$.
  \item \texttt{find} -- dla każdego odwiedzonego węzła:
        \begin{itemize}
           \item odczyt \texttt{parent} -- $O(1)$,
           \item zapis nowego \texttt{parent} --
                 \emph{co najwyżej} $\log^{\!*}\!n$ razy w~całej
                 sekwencji operacji.
        \end{itemize}
\end{itemize}

Przy $n$ elementach łączny koszt wszystkich zmian rodzica to
$O\!\bigl(n\log^{\!*}\!n\bigr)$,
a pozostałe odczyty i~\texttt{union}-y mieszczą się w $O(n)$.
Stąd amortyzowany koszt sekwencji $m=O(n)$ operacji wynosi
\[
  O\!\bigl(n\log^{\!*}\!n\bigr).
\]

\end{document}
