\documentclass[11pt,a4paper]{article}
\usepackage[utf8]{inputenc}
\usepackage[T1]{fontenc}
\usepackage[polish]{babel}
\usepackage{amsmath,amssymb}
\usepackage{algorithm}
\usepackage{algpseudocode}
\usepackage{geometry}
\geometry{margin=2.5cm}

\begin{document}

\begin{center}
\Large Dawid Pawliczek\\
Lista 6, Zadanie 1
\end{center}

\bigskip
%%%%%%%%%%%%%%%%%%%%%%%%%%%%%%%%%%%%%%%%%%%%%%%%%%%%%%%%%%%%%%%%%%%%
\section*{Treść}

Zaproponować \emph{nierekurencyjną} wersję sortowania Quick\-sort, która

\begin{itemize}
\item działa \emph{w miejscu} — poza tablicą $A[1\dots n]$ używa tylko
      stałej liczby zmiennych typu~\texttt{int},
\item wykonuje co najwyżej stały czynnik więcej operacji
      niż standardowa wersja rekurencyjna.
\end{itemize}

%%%%%%%%%%%%%%%%%%%%%%%%%%%%%%%%%%%%%%%%%%%%%%%%%%%%%%%%%%%%%%%%%%%%
\section*{Idea}

Klasyczny Quick\-sort zapisuje podzadania na stosie.
Obserwacja:

\begin{quote}
\emph{Po każdej operacji \textsc{partition}
 należy najpierw posortować \textbf{mniejszy} z dwóch
 powstałych przedziałów, a większy „odłożyć na później”.}
\end{quote}

Zamiast stosu większy przedział zakodujemy
\emph{bezpośrednio w tablicy} — przestawiając element,
który bezpośrednio \emph{za} pivotem wyróżnia jego prawą granicę.
Dwa wartowniki (wartości $+\infty$) na końcu tablicy
gwarantują, że wyjście z jednego przedziału
zawsze zostanie wykryte liniowym skanem.

%%%%%%%%%%%%%%%%%%%%%%%%%%%%%%%%%%%%%%%%%%%%%%%%%%%%%%%%%%%%%%%%%%%%
\section*{Algorytm}

\begin{algorithm}[H]
\caption{\textsc{IterQuickSort}}
\begin{algorithmic}[1]
\Require{$A[1\dots n]$ (elementy różne);\\
         $+\infty$ większe niż każdy $A[i]$ jest zapisywalne}
\Ensure{$A$ — posortowana rosnąco}
\State $A[n+1]\gets +\infty$;\;$A[n+2]\gets +\infty$
\State $l\gets1$;\;$r\gets n$                     \Comment{aktualny przedział}
\State $M\gets16$                                 \Comment{granica dla Insertion Sort}
\While{$l<r$}
   \While{$r-l+1>M$}                              \Comment{duży przedział — dziel}
      \State $(p)\gets$\textsc{HoarePartition}($A,l,r$)  \Comment{$A[p]$ = pivot}
      \If{$p-l<r-p$}                              \Comment{lewa strona mniejsza}
         \State $\text{swap}(A[p+1],A[r+1])$      \Comment{zapisz większy przedział}
         \State $r\gets p-1$                      \Comment{sortuj _mniejszy_ teraz}
      \Else                                       \Comment{prawa strona mniejsza}
         \State $\text{swap}(A[p-1],A[l-1])$
         \State $l\gets p+1$
      \EndIf
   \EndWhile
   \State \textsc{InsertionSort}$(A,l,r)$          \Comment{krótki przedział}
   \State                                          %%% „odczytaj” kolejny
         $l\gets r+2$;\;$r\gets l$                 \Comment{pomiń wartownik}
   \While{$A[r+1]<A[l-1]$}                        \Comment{szukaj kolejnej granicy}
         \State $r\gets r+1$
   \EndWhile
\EndWhile
\end{algorithmic}
\end{algorithm}

\paragraph{Używana pamięć.}
Stałe zmienne $l,r,p,M$,
dwa wartowniki i pivot wewnątrz tablicy.
Brak dynamicznego stosu — warunek „in place” spełniony.

%%%%%%%%%%%%%%%%%%%%%%%%%%%%%%%%%%%%%%%%%%%%%%%%%%%%%%%%%%%%%%%%%%%%
\section*{Poprawność}

\begin{enumerate}
\item \textbf{Działanie \textsc{partition}.}
      Funkcja Hoare’a rozdziela elementy na
      \(\;A[l..p-1]\!<\!A[p]\!<\!A[p+1..r]\).
\item \textbf{Wybór kolejności.}
      Zawsze sortujemy \emph{mniejszy} fragment
      (\(\min(p-l,\,r-p)\)) – gwarantuje to,
      że rozmiar „bieżącego” przedziału
      maleje co najwyżej o~połowę,
      a więc liczba aktywnych przedziałów
      w tym samym czasie nie przekracza~2.
\item \textbf{Kodowanie większego fragmentu.}
      Element tuż za pivotem zamieniamy
      z pierwszym wartownikiem $(+\infty)$.
      Dzięki temu liniowy skan od pivot+1
      napotyka pierwszy element $>\!$pivot
      dokładnie w~miejscu, gdzie zaczyna się
      odłożony przedział i jego granice są odzyskiwane.
\item \textbf{Terminacja.}
      Każdy element może zostać pivotem
      najwyżej raz; dla podtablic
      nie większych niż $M$
      używamy sortowania przez wstawianie.
      Algorytm kończy, gdy \(l\ge r\)
      (całość posortowana).
\end{enumerate}

%%%%%%%%%%%%%%%%%%%%%%%%%%%%%%%%%%%%%%%%%%%%%%%%%%%%%%%%%%%%%%%%%%%%
\section*{Złożoność}

\begin{itemize}
\item \textbf{Czas:}
      identyczny jak klasycznego Quick\-sorta
      z tą samą procedurą wyboru pivota.
      W średnim przypadku
      \(
        T(n)=\Theta(n\log n)
      \),
      w najgorszym \(\Theta(n^2)\);
      stała ukryta w~symbolu $\Theta$
      nie rośnie — zamiana stosu
      na kodowanie granic kosztuje $O(1)$
      na każdą wywołaną \textsc{partition}.
\item \textbf{Pamięć:}
      dokładnie \(\;O(1)\) słów \texttt{int}
      poza tablicą $A$.
\end{itemize}

\end{document}
