\documentclass[11pt,a4paper]{article}
\usepackage[utf8]{inputenc}
\usepackage[T1]{fontenc}
\usepackage[polish]{babel}
\usepackage{amsmath,amssymb}
\usepackage{algorithm}
\usepackage{algpseudocode}
\usepackage{geometry}
\geometry{margin=2.5cm}

\begin{document}

\begin{center}
\Large Dawid Pawliczek\\
Lista 1, Zadanie 4
\end{center}

\bigskip

\section*{Treść zadania}

\noindent
\emph{Nadsłowem} zbioru słów $\{w_1,w_2,\dots,w_k\}$ nazywamy dowolne
słowo $W$ takie, że
\(
  \forall_{i=1,\dots,k}\; w_i
\)
jest pod­słowem $W$.
Należy napisać algorytm, który dla danego zbioru \emph{słów
dwuliterowych} (tzn.\ $|w_i|=2$ dla każdego $i$) znajduje
\emph{najkrótsze} jego nadsłowo
(lub dowolne jedno z najkrótszych, gdy nie jest ono jednoznaczne).

\section*{Model grafowy}

Każde słowo $w_i=a_ib_i$ interpre­tujemy jako skierowaną krawędź
$a_i\!\to b_i$ w grafie $G=(V,E)$, gdzie $V$ jest zbiorem wszystkich
liter występujących w wejściu.
Słowo $W=v_0v_1\dots v_\ell$ jest nadsłowem wtedy
i tylko wtedy, gdy przebiegając kolejne krawędzie
$v_0\!\to v_1,\;v_1\!\to v_2,\dots$
odwiedzamy \emph{każdą} krawędź z~$E$
co najmniej raz.
Szukamy zatem ścieżki (lub cyklu) Eulera w~$G$,
przy czym powtórne przejścia po krawędziach wydłużają wynik
– interesuje nas najkrótsza ścieżka obejmująca wszystkie krawędzie.

\section*{Redukcja do cyklu Eulera}

\begin{description}
  \item[Spójność.]
        Jeśli $G$ nie jest spójny, konstruujemy rozwiązanie osobno
        dla każdej składowej, a następnie łączymy otrzymane słowa
        w~dowolnej kolejności; ich konkatenacja wciąż zawiera wszystkie
        wejściowe krawędzie.
  \item[Niespełnienie warunku Eulera.]
        W spójnej składowej
        istnieje ścieżka Eulera
        wtedy i tylko wtedy, gdy \\
        \centerline{$\bigl|\{v: \text{outdeg}(v)=\text{indeg}(v)+1\}\bigr|=%
                   \bigl|\{v: \text{indeg}(v)=\text{outdeg}(v)+1\}\bigr|=1$,}
        a~dla cyklu Eulera gdy
        $\text{indeg}(v)=\text{outdeg}(v)$ dla każdego~$v$.
        Aby ujednolicić przypadki,
        dodajemy nowy wierzchołek $\varepsilon$ (pusta litera)
        oraz tyle krawędzi $v\!\to\varepsilon$ lub $\varepsilon\!\to v$,
        ile potrzeba, by zrównoważyć stopnie wszystkich
        wierzchołków – wtedy cały graf \emph{zawsze} posiada
        cykl Eulera.  Litery $\varepsilon$ pomijamy w końcowym słowie.
\end{description}

\section*{Algorytm}

\begin{algorithm}[H]
\caption{\textsc{Superword}$(\{w_1,\dots,w_k\})$}
\begin{algorithmic}[1]
\State $G=(V,E)\gets(\varnothing,\varnothing)$
\ForAll{$w_i=a_ib_i$}
  \State $V\gets V\cup\{a_i,b_i\}$;\quad $E\gets E\cup\{a_i\!\to b_i\}$
\EndFor
\State $W\gets\text{``''}$
\ForAll{spójna składowa $G'=(V',E')$ grafu $G$}
  \If{$G'$ spełnia warunek cyklu Eulera}
     \State $P\gets$ cykl Eulera w $G'$ (np.\ alg.~Hierholzera)
  \Else
     \State dodaj nowy wierzchołek $\varepsilon$ do $V'$
     \State wyrównaj stopnie, dodając krawędzie
            $\varepsilon\!\leftrightarrow v$ w~miarę potrzeby
     \State $P\gets$ cykl Eulera w~tak rozszerzonym $G'$
     \State usuń z~$P$ wszystkie wystąpienia $\varepsilon$
  \EndIf
  \State $W\gets W\,+$ litery ścieżki $P$
\EndFor
\State \Return{$W$}
\end{algorithmic}
\end{algorithm}

\section*{Złożoność}

Niech $n=|V|$, $m=|E|=k$.
Konstrukcja grafu i obliczenie stopni trwa $O(n+m)$.
Dla każdej składowej wywołujemy algorytm Hierholzera
$O(n'+m')$, co łącznie daje~$O(n+m)$.
Pamięć pomocnicza: $O(n+m)$.

\section*{Poprawność -- szkic}

\begin{enumerate}
  \item \textbf{Pokrycie krawędzi.} Każde słowo powstaje ze
        ścieżki/cyklu Eulera, więc zawiera wszystkie krawędzie wejścia,
        zatem jest nadsłowem.
  \item \textbf{Minimalność.} Algorytm Hierholzera przechodzi każdą
        krawędź dokładnie raz; ewentualne krawędzie do $\varepsilon$
        odpowiadają dokładnie niezbilansowanym przejściom, które i tak
        musiałyby zostać pokryte przez dodatkowe litery w~dowolnym
        nadsłowie.  Po pominięciu $\varepsilon$ otrzymujemy zatem
        najkrótsze możliwe słowo.
\end{enumerate}

\end{document}
