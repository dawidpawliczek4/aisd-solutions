\documentclass[a4paper,12pt]{article}
\usepackage[utf8]{inputenc}
\usepackage[T1]{fontenc}
\usepackage[polish]{babel}
\usepackage{amsmath,amssymb}
\usepackage[ruled,vlined,linesnumbered]{algorithm2e}

\title{Zadanie 2 z Listy 1 \\ \large{Dawid Pawliczek}}
\author{}
\date{}

\begin{document}
\maketitle

\section*{Treść zadania}
Mamy dane $k$ list $L_1, L_2, \dots, L_k$, każda uporządkowana niemalejąco, zawierająca liczby całkowite. 
Należy znaleźć przedział $[a, a + r]$ o najmniejszej możliwej długości $r$, taki że w tym przedziale znajduje się co najmniej jedna wartość z każdej listy $L_i$.

\medskip
\noindent \textbf{Ograniczenia:}
\begin{itemize}
  \item Nie wolno modyfikować list $L_i$ (np. scalać ich w jedną wspólną tablicę).
  \item Algorytm powinien być oszczędny pamięciowo i możliwie jak najszybszy.
\end{itemize}

\section*{Idea rozwiązania}
Obserwacja — stan rozwiązania można opisać za pomocą wskaźników
\[
  p = (p_1, p_2, \dots, p_k),
\]
gdzie każdy wskaźnik $p_i$ wskazuje na pewien element listy $L_i$. Dla danego stanu $p$ rozważamy przedział
\[
  \bigl[\min_{1 \le i \le k} L_i[p_i],\;\; \max_{1 \le i \le k} L_i[p_i]\bigr].
\]
Chcemy znaleźć taki stan $p$, dla którego różnica
\[
  \max_i L_i[p_i] - \min_i L_i[p_i]
\]
jest minimalna.

Przesuwanie wskaźników:
\begin{itemize}
  \item Z założenia przesunięcie wskaźnika wskazującego na największy element \emph{zawsze} może tylko zwiększyć wartość maksymalną—nie ma więc sensu.
  \item Optymalnie przesuwamy wskaźnik wskazujący na bieżąco najmniejszy element.
\end{itemize}

Aby w każdej iteracji w $O(\log k)$ odnaleźć, który wskaźnik wskazuje aktualnie najmniejszy element, używamy kopca typu „min-heap” o rozmiarze $k$.

\section*{Algorytm w pseudokodzie}
\begin{algorithm}[H]
  \SetKwInOut{Input}{wejście}\SetKwInOut{Output}{wyjście}
  \Input{Uporządkowane niemalejąco listy $L_1,\dots,L_k$.}
  \Output{Przedział $[a,b]$ minimalizujący długość $b - a$, który pokrywa po jednym elemencie z każdej listy.}
  \BlankLine
  % inicjalizacja
  $M \leftarrow -\infty$, $m \leftarrow +\infty$\;
  niech $H$ będzie pustym kopcem typu \texttt{(wartość, indeks\_listy, indeks\_w\_liście)}\;
  \For{$i \leftarrow 1$ \KwTo $k$}{
    $v \leftarrow L_i[0]$\;
    $M \leftarrow \max(M, v)$\;
    $m \leftarrow \min(m, v)$\;
    $H.\text{push}(\,(v,i,0)\,)$\;
  }
  % główna pętla
  \While{kopiec $H$ nie jest pusty}{
    $(v,i,j) \leftarrow H.\text{popMin}()$\;
    \uIf{$M - m < \text{najlepsza\_długość}$}{
      \text{aktualizuj najlepszy przedział }$[a,b] \leftarrow [m,M]$\;
    }
    j \;\texttt{++}\;
    \If{ $j \ge |L_i|$ }{
      \textbf{break}\;  % nie można dalej posuwać wskaźnika
    }
    $v' \leftarrow L_i[j]$\;
    $M \leftarrow \max(M, v')$\;
    % m zostanie ustalone przez najmniejszy element w kopcu:
    $H.\text{push}(\,(v',i,j)\,)$\;
    $m \leftarrow H.\text{minValue}()$\;
  }
  \Return $[a,b]$\;
  \caption{Znalezienie najmniejszego przedziału zawierającego po jednym elemencie z każdej z $k$ list.}
\end{algorithm}

\section*{Złożoność obliczeniowa}
\begin{itemize}
  \item Każdy ze $n = \sum_i |L_i|$ elementów jest wstawiany i usuwany z kopca dokładnie raz.
  \item Operacja wstawienia i usunięcia z kopca o rozmiarze $k$ kosztuje $O(\log k)$.
  \item Całkowity czas: $\displaystyle O\bigl(n \log k\bigr)$.
  \item Dodatkowa pamięć: $O(k)$ na kopiec i kilka zmiennych pomocniczych.
\end{itemize}

\end{document}
