\documentclass[11pt,a4paper]{article}
\usepackage[utf8]{inputenc}
\usepackage[T1]{fontenc}
\usepackage[polish]{babel}
\usepackage{amsmath,amssymb}
\usepackage{algorithm}
\usepackage{algpseudocode}
\usepackage{geometry}
\geometry{margin=2.5cm}

\begin{document}

\begin{center}
\Large Lista 1, zadanie 5\\

\end{center}

\bigskip
%%%%%%%%%%%%%%%%%%%%%%%%%%%%%%%%%%%%%%%%%%%%%%%%%%%%%%%%%%%%%%%%%%%%
\section*{Założenia i notacja}

Dany jest skierowany graf acykliczny
\(G=(V,E)\), \(|V|=n\), \(|E|=m\).
Długość ścieżki liczona jest w~liczbie krawędzi.
Zakładamy, że graf jest przedstawiony listami
następników.

%%%%%%%%%%%%%%%%%%%%%%%%%%%%%%%%%%%%%%%%%%%%%%%%%%%%%%%%%%%%%%%%%%%%
\section{Liczenie \emph{długości} najdłuższej ścieżki}

\subsection*{Idea}
W DAG-u istnieje uporządkowanie topologiczne
\(T=(v_1,\dots,v_n)\), w którym każda krawędź
idzie „w prawo”.
Przechodząc w tej kolejności
można dynamicznie uzupełniać wartość
\[
\text{dp}[v]=\text{długość najdłuższej ścieżki
            kończącej się w }v .
\]

\subsection*{Algorytm}

\begin{algorithm}[H]
\caption{\textsc{LongestPathLength}}
\begin{algorithmic}[1]
\Require{$G=(V,E)$ — DAG}
\Ensure{długość najdłuższej ścieżki w $G$}
\State $T\gets$ topologiczne uporządkowanie
       (np.\ algorytm Kahna)
\State \textbf{for} $v\in V$: $\text{dp}[v]\gets 0$
\For{$v$ w kolejności $T$}               \Comment{każdy wierzchołek raz}
   \ForAll{$w$ następniki $v$}
      \If{$\text{dp}[v]+1>\text{dp}[w]$}
         \State $\text{dp}[w]\gets\text{dp}[v]+1$
      \EndIf
   \EndFor
\EndFor
\State \Return{$\max_{v\in V}\text{dp}[v]$}
\end{algorithmic}
\end{algorithm}

\paragraph{Złożoność.}
Uporządkowanie topologiczne $O(n+m)$,
pętla również $O(n+m)$.
Pamięć $O(n)$.

%%%%%%%%%%%%%%%%%%%%%%%%%%%%%%%%%%%%%%%%%%%%%%%%%%%%%%%%%%%%%%%%%%%%
\section{Wypisywanie jednej maksymalnej ścieżki}

\subsection*{Modyfikacja}
Wystarczy zapamiętywać
poprzednik, z którego
uzyskano najlepszą wartość \texttt{dp}.
\[
\text{parent}[w]=
\begin{cases}
v, &\text{gdy } \text{dp}[v]+1>\text{dp}[w],\\
\texttt{nil}, &\text{jeśli brak aktualizacji.}
\end{cases}
\]

\begin{algorithm}[H]
\caption{\textsc{LongestPath}}
\begin{algorithmic}[1]
\Require{$G=(V,E)$}
\Ensure{jedna najdłuższa ścieżka w kolejności źródło→cel}
\State $T\gets$ topologiczne uporządkowanie
\For{$v\in V$} $\text{dp}[v]\gets0$,\; $\text{parent}[v]\gets\texttt{nil}$ \EndFor
\For{$v$ w $T$}
   \ForAll{$w\in\text{succ}(v)$}
      \If{$\text{dp}[v]+1>\text{dp}[w]$}
         \State $\text{dp}[w]\gets\text{dp}[v]+1$
         \State $\text{parent}[w]\gets v$
      \EndIf
   \EndFor
\EndFor
\State $u\gets\arg\max_{v}\text{dp}[v]$      \Comment{koniec ścieżki}
\State ścieżka $\gets[\,]$
\While{$u\neq\texttt{nil}$}
   \State wstaw $u$ na początek ścieżki
   \State $u\gets\text{parent}[u]$
\EndWhile
\State \Return{ścieżka}
\end{algorithmic}
\end{algorithm}

\paragraph{Złożoność.}
Niezmieniona: $O(n+m)$ czasu, $O(n)$ pamięci  
(dodatkowa tablica \texttt{parent}).

%%%%%%%%%%%%%%%%%%%%%%%%%%%%%%%%%%%%%%%%%%%%%%%%%%%%%%%%%%%%%%%%%%%%
\section*{Poprawność (krótko)}

Przetwarzając wierzchołki w kolejności topologicznej
mamy pewność, że dla każdej krawędzi
\(v\!\to\!w\) wartość \(\text{dp}[v]\) jest finalna
w momencie aktualizowania \(\text{dp}[w]\);
otrzymujemy największą możliwą długość ścieżki
kończącej się w $w$.  
Indukcja po kolejnych wierzchołkach
dowodzi poprawności tablicy \texttt{dp},
a śledzenie \texttt{parent}
od elementu o maksymalnym \texttt{dp}
odtwarza ścieżkę o~tej właśnie długości.
\hfill$\square$

\end{document}
