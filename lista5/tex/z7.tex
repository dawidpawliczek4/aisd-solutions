\documentclass[11pt,a4paper]{article}
\usepackage[utf8]{inputenc}
\usepackage[T1]{fontenc}
\usepackage[polish]{babel}
\usepackage{amsmath,amssymb}
\usepackage{geometry}
\geometry{margin=2.5cm}

\begin{document}

\begin{center}
\Large Dawid Pawliczek\\
Lista 5, Zadanie 7
\end{center}

\bigskip
%%%%%%%%%%%%%%%%%%%%%%%%%%%%%%%%%%%%%%%%%%%%%%%%%%%%%%%%%%%%%%%%%%%%
\section*{Teza}

W modelu drzew decyzyjnych każde scalanie
\emph{dwóch} posortowanych ciągów długości~$n$
wymaga w~najgorszym przypadku
co~najmniej $2n-1$ porównań.

%%%%%%%%%%%%%%%%%%%%%%%%%%%%%%%%%%%%%%%%%%%%%%%%%%%%%%%%%%%%%%%%%%%%
\section*{Gra z adwersarzem}

Oznaczmy elementy pierwszego ciągu
\(a_1<a_2<\dots<a_n\),
drugiego
\(b_1<b_2<\dots<b_n\).

Adwersarz ogranicza przestrzeń wejść do
\[
  \mathcal X=\{X_0,X_1,\dots,X_{2n-1}\},
\]
gdzie
\[
\begin{aligned}
X_0      &= a_1,b_1,a_2,b_2,\dots,a_n,b_n,\\
X_{2k-1} &= X_0 \text{ z zamianą } b_k\!\leftrightarrow a_k,\\
X_{2k}   &= X_0 \text{ z zamianą } b_k\!\leftrightarrow a_{k+1}
           \quad(1\le k\le n-1).
\end{aligned}
\]
Każdy $X_i$ jest poprawnym wynikiem scalania
i wszystkie są różne, więc algorytm
musi ostatecznie rozróżnić wszystkie $2n$ kandydatów.

\paragraph{Odpowiedzi adwersarza.}
Algorytm pyta tylko o relacje \(\,a_i\,\text{vs}\,b_j\).

\begin{itemize}
\item $|i-j|>1$: adwersarz odpowiada $a_i<b_j$
      (nie eliminuje żadnego $X_k$).
\item $i=j$:      adwersarz odpowiada $b_i<a_i$,
      eliminując wyłącznie $X_{2i-1}$.
\item $i=j+1$:    odpowiada $a_i<b_j$,
      eliminując wyłącznie $X_{2j}$.
\end{itemize}

Każde pytanie usuwa z~$\mathcal X$
\emph{co najwyżej jeden} ciąg;
aby pozostał jeden, potrzeba co najmniej $2n-1$ pytań.

%%%%%%%%%%%%%%%%%%%%%%%%%%%%%%%%%%%%%%%%%%%%%%%%%%%%%%%%%%%%%%%%%%%%
\section*{Przykład dla $n=3$}

\subsection*{Zestaw kandydatów}

\[
\begin{aligned}
X_0 &= a_1,\,b_1,\,a_2,\,b_2,\,a_3,\,b_3,\\
X_1 &= b_1,\,a_1,\,a_2,\,b_2,\,a_3,\,b_3,\\
X_2 &= a_1,\,a_2,\,b_1,\,b_2,\,a_3,\,b_3,\\
X_3 &= a_1,\,b_1,\,b_2,\,a_2,\,a_3,\,b_3,\\
X_4 &= a_1,\,b_1,\,a_2,\,a_3,\,b_2,\,b_3,\\
X_5 &= a_1,\,b_1,\,a_2,\,b_2,\,b_3,\,a_3.
\end{aligned}
\]

\subsection*{Jak adwersarz odpowiada na pytania}

\renewcommand{\arraystretch}{1.25}
\begin{center}
\begin{tabular}{|c|l|c|c|l|}
\hline
\textbf{Typ} & \multicolumn{1}{c|}{\textbf{Przykład pytania}}
             & \textbf{Odp.} & \textbf{Elim. $X_k$} & \multicolumn{1}{c|}{\textbf{Dlaczego}}\\\hline
$|i-j|>1$ &
$a_3 \;?\; b_1$ &
$a_3<b_1$ &
--- &
w~żadnym $X_k$ nie ma $b_1<a_3$      \\\hline
$i=j$ &
$a_2 \;?\; b_2$ &
$b_2<a_2$ &
$X_3$ &
bo tylko w $X_3$: $a_2<b_2$          \\\hline
$i=j+1$ &
$a_2 \;?\; b_1$ &
$a_2<b_1$ &
$X_2$ &
bo tylko w $X_2$: $b_1<a_2$          \\\hline
\end{tabular}
\end{center}

W każdym wierszu najwyżej \emph{jeden} kandydat
staje się sprzeczny z odpowiedzią,
zgodnie z ogólnym dowodem.

%%%%%%%%%%%%%%%%%%%%%%%%%%%%%%%%%%%%%%%%%%%%%%%%%%%%%%%%%%%%%%%%%%%%
\section*{Wniosek}

Aby zredukować $\lvert\mathcal X\rvert=2n$ kandydatów do jednego,
algorytm musi zadać przynajmniej $2n-1$ pytań,
co dowodzi dolnej granicy
\[
\boxed{\;2n-1\;}
\]
porównań dla scalania dwóch $n$‐elementowych ciągów.
\hfill$\square$

\end{document}
