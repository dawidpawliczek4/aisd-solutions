\documentclass[11pt,a4paper]{article}
\usepackage[utf8]{inputenc}
\usepackage[T1]{fontenc}
\usepackage[polish]{babel}
\usepackage{amsmath,amssymb,amsthm}
\usepackage{geometry}
\geometry{margin=2.5cm}

\begin{document}

\begin{center}
\Large Dawid Pawliczek\\
Lista 5, Zadanie 2
\end{center}

\bigskip
%%%%%%%%%%%%%%%%%%%%%%%%%%%%%%%%%%%%%%%%%%%%%%%%%%%%%%%%%%%%%%%%%%%%
\section*{Teza}

W modelu \emph{liniowych drzew decyzyjnych}\footnote{
Każdy test ma postać
$A_1x_1+\dots+A_nx_n\;\;\mathop{\bowtie}\;0$,
gdzie $\bowtie\in\{<,=,>\}$, zaś w liściach
zwracane są ciągi indeksów punktów tworzących otoczkę wypukłą.}
każdy algorytm obliczający otoczkę wypukłą
$n$ punktów na płaszczyźnie
musi w~najgorszym przypadku wykonywać
$\Omega(n\log n)$ porównań.

%%%%%%%%%%%%%%%%%%%%%%%%%%%%%%%%%%%%%%%%%%%%%%%%%%%%%%%%%%%%%%%%%%%%
\section*{Konstrukcja instancji}

Umieszczamy $2n$ par różnobiegunowych punktów
na~jednostkowym okręgu, co drugiemu
nadając etykietę \emph{czarny}, a pozostałym – \emph{biały}.
Czarne punkty pozostają w ustalonych pozycjach,
natomiast białe będą \emph{permuto­wane}
wzdłuż swoich miejsc „na zegarze”
(rysunek pomijamy).

Łącznie istnieje $(n!)$ permutacji białych punktów;
poniżej pokażemy, że \textbf{każda z nich
prowadzi do innej otoczki wypukłej}.

%%%%%%%%%%%%%%%%%%%%%%%%%%%%%%%%%%%%%%%%%%%%%%%%%%%%%%%%%%%%%%%%%%%%
\section*{Dlaczego permutacje różnią odpowiedź}

Niech $p,q$ będą dwoma kolejnymi białymi punktami
na otoczce dla pewnej permutacji $\pi$.
Jeśli w permutacji $\pi'$ punkty te zamienimy miejscami,
to w~miarę zbliżania się do siebie
(przy zachowaniu okręgu)
krawędź $pq$ skraca się i gdy punkty zetkną się,
odcinek $pq$ znajdzie się wewnątrz
otoczki tworzonej przez pozostałe wierzchołki.
Stąd po dowolnej zamianie kolejności białych punktów
zestaw \emph{wierzchołków} otoczki zmienia się,
a więc procedura musi rozróżnić
każdą z $(n!)$ permutacji.

%%%%%%%%%%%%%%%%%%%%%%%%%%%%%%%%%%%%%%%%%%%%%%%%%%%%%%%%%%%%%%%%%%%%
\section*{Liczenie liści}

W liniowym drzewie decyzyjnym każdy
liść odpowiada \emph{jednej} możliwej odpowiedzi.
Skoro odpowiedzi jest $n!$, drzewo ma
co najmniej $n!$ liści.
Drzewo binarne o $L$ liściach ma wysokość
$\;\ge\lceil\log_2 L\rceil$,
więc wysokość naszego drzewa to
\[
  \Omega\!\bigl(\log (n!)\bigr)
  \;=\;
  \Omega(n\log n).
\]

%%%%%%%%%%%%%%%%%%%%%%%%%%%%%%%%%%%%%%%%%%%%%%%%%%%%%%%%%%%%%%%%%%%%
\section*{Konkluzja}

Każde liniowe drzewo decyzyjne
rozwiązujące problem otoczki wypukłej
musi mieć wysokość 
$\Omega(n\log n)$,
co wyznacza dolną granicę
\(
  \Omega(n\log n)
\)
na liczbę niezbędnych porównań.
\hfill$\square$

%%%%%%%%%%%%%%%%%%%%%%%%%%%%%%%%%%%%%%%%%%%%%%%%%%%%%%%%%%%%%%%%%%%%
\paragraph{Uwaga.}
Klasyczny dowód można też oprzeć
na redukcji z~sortowania
(umieszczamy punkty $(x_i,x_i^2)$ na paraboli);
zaprezentowana argumentacja okręgowa
nie odwołuje się do operacji nieliniowych,
tym samym bezpośrednio wpisuje się
w model liniowych testów.
\end{document}
