\documentclass[11pt,a4paper]{article}
\usepackage[utf8]{inputenc}
\usepackage[T1]{fontenc}
\usepackage[polish]{babel}
\usepackage{amsmath,amssymb}
\usepackage{geometry}
\geometry{margin=2.5cm}

\begin{document}

\begin{center}
\Large Dawid Pawliczek\\
Lista 5, Zadanie 8
\end{center}

\bigskip
%%%%%%%%%%%%%%%%%%%%%%%%%%%%%%%%%%%%%%%%%%%%%%%%%%%%%%%%%%%%%%%%%%%%
\section*{Teza}

Aby wyznaczyć największy \emph{i} drugi co do wielkości element
w~$n$-elementowym zbiorze, w~modelu drzew decyzyjnych
\[
  \boxed{\,n+\lceil\log_2 n\rceil-2\,}
\]
porównań \emph{wystarcza} i \emph{jest konieczne}.

%%%%%%%%%%%%%%%%%%%%%%%%%%%%%%%%%%%%%%%%%%%%%%%%%%%%%%%%%%%%%%%%%%%%
\section{Górna granica —~$n+\lceil\log n\rceil-2$ wystarcza}

\subsection*{Algorytm turniejowy}

\begin{enumerate}
\item
Parujemy elementy i porównujemy je.
Zwycięzcy przechodzą do następnej rundy,
przegrani wypadają.
\item
Powtarzamy turniej, aż zostanie jeden \textbf{maksymalny}
— wymaga to $n-1$ porównań.
\item
Największy element pokonał bezpośrednio
dokładnie tyle elementów, ile wynosi
wysokość drzewa turnieju:\;
$h=\lceil\log_2 n\rceil$.
\item
Drugi wynik znajdujemy, porównując między sobą
pokonanych przez maksimum — potrzeba
$h-1=\lceil\log_2 n\rceil-1$ dodatkowych testów.
\end{enumerate}

\[
  (n-1)+\bigl(\lceil\log_2 n\rceil-1\bigr)
  \;=\;
  n+\lceil\log_2 n\rceil-2.
\]

%%%%%%%%%%%%%%%%%%%%%%%%%%%%%%%%%%%%%%%%%%%%%%%%%%%%%%%%%%%%%%%%%%%%
\section{Dolna granica —~$n+\lceil\log n\rceil-2$ jest konieczne}

\subsection*{Gra z adwersarzem}

Dla każdego elementu $x$ adwersarz prowadzi listę
„wiadomo, że~\,$x$ $>$ te elementy”.

\paragraph{Reguła odpowiedzi.}
Na zapytanie „$x\;?\;y$”
adwersarz porównuje długości list $L(x)$ i~$L(y)$
i zawsze ogłasza większym ten, którego lista jest dłuższa
(remis rozstrzygany dowolnie).
Po odpowiedzi listę zwycięzcy rozszerza o przegranego
oraz wszystkie jego zapisane elementy.

\subsection*{Konsekwencje}

\begin{itemize}
\item \
Każde porównanie może co najwyżej
\emph{podwoić} rozmiar listy zwycięzcy.
\item \
Aby jakiś element wskazać jako \emph{największy},
jego lista musi urosnąć do $n$ pozycji,
czyli potrzeba co najmniej
$\lceil\log_2 n\rceil$ zwycięstw $\;(\ge n-1$ porównań
na całym zbiorze).
\item \
Elementy zapisane w liście maksimum
to \emph{jedyni kandydaci na drugie miejsce}.
Tych kandydatów jest co najmniej
$\lceil\log_2 n\rceil$,
więc rozróżnienie ich wymaga dodatkowo
co najmniej
$\lceil\log_2 n\rceil-1$ porównań.
\end{itemize}

\[
  (n-1)
  +\bigl(\lceil\log_2 n\rceil-1\bigr)
  \;=\;
  n+\lceil\log_2 n\rceil-2
\]
— mniej się nie da.

%%%%%%%%%%%%%%%%%%%%%%%%%%%%%%%%%%%%%%%%%%%%%%%%%%%%%%%%%%%%%%%%%%%%
\section*{Wniosek}

Zarówno górna, jak i dolna granica
pokrywają się z wartością
\(
  n+\lceil\log_2 n\rceil-2,
\)
więc jest to dokładna liczba
porównań potrzebnych i wystarczających
do wyznaczenia największego oraz
drugiego największego elementu.
\hfill$\square$

\end{document}
